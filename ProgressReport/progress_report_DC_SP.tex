\documentclass[12pt]{article}
\usepackage[a4paper,margin=1in]{geometry}
\usepackage[utf8]{inputenc}
\usepackage[english]{babel}
\usepackage{csquotes}
\usepackage{setspace}

\usepackage[backend=biber,style=numeric]{biblatex}
\addbibresource{bibliography.bib}

\title{Progress Report}
\author{Daniel Cloutier \and Sagar Patel}
\date{\today}
\doublespacing

\begin{document}
    \begin{singlespace}
        \maketitle 
    \end{singlespace}

    \section{Problem Statement}

    In this era, there is a massive amount of data collected in a large variety of different topics, of which lots of information can be extracted. The issue here comes with the fact that we cannot simply extract the information by visual inspection. There is currently too much data. 
    
    One such domain that contains a large amount of data is the game of chess. Many players like to study chess games to try and improve their play, but it's very hard to find games that are relevant or interesting to learn from amidst the plethora of games to choose from. Although there are many famous known games such as the Opera game of Paul Morphy vs Duke Karl or game 6 from the 1972 world championship match between Robert Fischer vs Boris Spassky \cite{chessgames}, these are usually kept as beginner studies. Also, the fact that they are already well known doesn't necessarily help us discover anything new. 
    
    On top of these numerous known games, surely there are many more games that you could learn from. In fact, the website lichess.org logged 68,027,862 games at all levels of play in September of 2020 alone \cite{lichessdb}. Some of these could be interesting to study, irrespective of the players ratings not being at the level of Grandmaster. All of this to say, there must be support from algorithms or machine learning techniques to be able to group similar games of interest together, or show games that lie far outside the norm, which could be interesting studies.

    \section{Problem Analysis}

    In order to tackle this problem, we must try and quantify what makes a game more or less similar to others and perhaps what other parameters might be used to determine whether a game would be worth studying or not. Luckily for us, all moves in a chess game can be represented in a general format called \textit{Portable Game Notation} (PGN), which displays moves in a human readable ASCII format using \textit{Standard Algebraic Notation} (SAN). This PGN contains information such as the date, time control, player names, player ratings as well as of course, the moves played during the game in SAN. SAN is used to describe which piece is moved ([Q]ueen, [R]ook, [K]ing, k[N]ight, [B]ishop, [P]awn/nothing) and to which square it is moved, using a system of a-h for horizontal position and 1-8 for vertical position, using the White player as the point of reference. We can also represent games using Forsyth-Edwards Notation (FEN) which describes a specific position in a chess game in one line of ASCII text \cite{pgnrules}.

    For example, we could define a total "distance" of a game as some sum of the differences between two moves, defining the distance as some function of piece values, distance moved on the board and whether a piece was capture or not.

    We must also take into consideration that the data we will be using will likely be the lichess games database, which contains games at all levels of play. This means some games will be of very poor quality while some will be of extremely high quality. In fact, some Grandmaster level players play chess online, or even stream their own games. This means that our data should be sorted, or at least quantified on more than one axis to more aptly determine whether a game that is an outlier is significantly more interesting than others, or is just a very poor quality game played by beginners.

    Finally, we must also look at the clock times. Oftentimes, chess games are played with different time controls, varying anywhere between 1 minute for all your moves, up to the official International Chess Federation (FIDE - Fédération Internationale des Échecs) time control of "90 minutes for the first 40 moves of the game, followed by 30 minutes for the rest of the game with an additional 30 seconds added per move starting move 1." \cite{fiderules}. This also can affect the quality of games, as a game that lasts a mere 2 minutes will likely be of far lesser quality than a game that lasted multiple hours. 

    \section{Literature Review}

    Based on this problem analysis we were able to find related literature that addressed some of these problems. Namely, we found a distance function that classifies a move as a 9-dimensional vector \cite{main}. Before we go into this however, they also define weights to each chess piece. Namely: 

    \begin{table}[ht]
        \begin{tabular}{|l|l|l|l|l|l|l|l|}
        \hline
            & K  & Q & R & B & N & P & NULL \\ \hline
        Weight & 12 & 9 & 5 & 3 & 2 & 1 & 0    \\ \hline
        \end{tabular}
    \end{table}

    With the dimensions of the vector being: 

    \begin{enumerate}
        \item x coordinate of the piece before move using SAN 
        \item y coordinate of the piece before move using SAN 
        \item x coordinate of the piece after move using SAN 
        \item y coordinate of the piece after move using SAN 
        \item x coordinate of the piece captured using SAN 
        \item y coordinate of the piece captured using SAN 
        \item weight of the piece before move 
        \item weight of the piece after move 
        \item weight of the piece captured
    \end{enumerate}

    Unsupervised:

        They used only 5000 GM games, this is good but we can do better. Lichess database has 1.5 Billion games available to us. Although they are at faster speeds. Standard FIDE clocks are 90+30, 30 minutes added at move 40. The slowest standard lichess game is 30+20. However, we can definitely filter our games because we have access to these. 

        The distance function. This is fundamentally good, but evaluating chess games on only one dimension to try and group or find outliers is not that telling. Why? Well:

        \begin{enumerate}
            \item Chess Openings (ECO Codes): Games with similar ECO codes will by default be similar. For example, ECO codes B20-B99 are all different variations of which the first moves are 1. e4 c5. This is relevant because if, for example, a game with opening B20 (Sicilian Defense) somehow has a similar distance than multiple games with code C53 (Giuoco Piano; 1. e4 e5 2. Nf3 Nc6 3. Bc4 Bc5 4. c3), that game could turn out to be interesting. 
            \item Player rating (ELO): With our data, games will be spread out over multiple different skill levels. Because of this, it is important to sort our games by skill level 
            \item Game Quality: It is possible to be able to sort our games by the "quality" of the game; that is we would define a function that looks at a move and determines whether it is good or bad. Either we can make our own, or we can use something existing. One thing to note is that we're changing the piece values so it is possible to at least attempt to make our own evaluation function if we want to.
            \item Game Clock: Some games also have not only the total time for the game, but also have the amount of time a player took per move (starting April 2017 on Lichess). This information can also be used to further classify games. For example, you may not want to look at games in Bullet format (1 minute total to make all moves) if you're trying to improve at a slower time format. But if you want to improve in that format specifically, this might actually be something of value.
        \end{enumerate}

    \section{Subproblems, Statement and Analysis}

    Right from the beginning of this project, we wanted to extract meaningful information from chess data in addition to implementing the research provided. Luckily we have access to over 1.5 billion records of chess games played by people all around the world. We believe there is valuable information to learn if we can extract it effectively. We decided to look deeper into the point values assigned to each chess piece and the circumstances can affect these values. During a chess match, each player can be assigned scores that are evaluated by various position features - most importantly the number of pieces on each side and further the positions, centralization, and mobility of each piece. By adjusting the individual weight of specific pieces on the board it allows us to emphasizes the difference in the importance of these pieces. The literature we have provided sampled their research using the standard "Reinfeld values" in which pawn=1, bishop=knight=3, rook=5, and queen=9 while the king is given an infinite value. While this is a good guide, chess is rarely that simple. The evaluation of the pieces can be changed due to many parameters such as pawns near the edges are worth less than those near the center, pawns close to promotion are worth far more, pieces controlling the center are worth more than average and trapped pieces are worthless, etc. Selecting effective weights for pieces allows us to estimate the player's positional advantage. For example, suppose we see a randomly chosen position in which White has a pawn advantage of 2 points. With a probability of close to 80\% , we can assert a White win. 

    What is the correct weight for chess pieces? Unfortunately, there is no right answer however, there are various famous algorithms that attempt to cover different strategies and positions. We intend to use these weighted values to find advantages that were not discovered by the researchers using simple Reinfeld values and uncover interesting strategies. 
    

    Subproblem: Not all games have position evaluations, so we will possibly have to do our own position evaluations if we want to have that as a parameter for the unsupervised algorithm. That or we can scrap that idea as it isn't really necessary considering player ELO ratings will be already a good indicator of the quality of games.    


    \section{Algorithmic Sketch, Illustration of Solution}

    Not sure what we would do here, but we can probably draw something pretty quick in paint (or figure out how to do graphs in \LaTeX) as our sketch. It probably isn't that complex\dots Probably a famous last words moment though.

    \printbibliography
    

\end{document}